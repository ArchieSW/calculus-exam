% Этот шаблон документа разработан в 2014 году
% Данилом Фёдоровых (danil@fedorovykh.ru) 
% для использования в курсе 
% <<Документы и презентации в \LaTeX>>, записанном НИУ ВШЭ
% для Coursera.org: http://coursera.org/course/latex .
% Исходная версия шаблона --- 
% https://www.writelatex.com/coursera/latex/3.2

\documentclass[a4paper,12pt]{article}

%%% Работа с русским языком
\usepackage{cmap}					% поиск в PDF
\usepackage{mathtext} 				% русские буквы в фомулах
\usepackage[T2A]{fontenc}			% кодировка
\usepackage[utf8]{inputenc}			% кодировка исходного текста
\usepackage[english,russian]{babel}	% локализация и переносы

%%% Дополнительная работа с математикой
\usepackage{amsmath,amsfonts,amssymb,amsthm,mathtools} % AMS
\usepackage{icomma} % "Умная" запятая: $0,2$ --- число, $0, 2$ --- перечисление

%% Номера формул
%\mathtoolsset{showonlyrefs=true} % Показывать номера только у тех формул, на которые есть \eqref{} в тексте.
%\usepackage{leqno} % Немуреация формул слева

%% Свои команды
\DeclareMathOperator{\sgn}{\mathop{sgn}}

%% Перенос знаков в формулах (по Львовскому)
\newcommand*{\hm}[1]{#1\nobreak\discretionary{}
{\hbox{$\mathsurround=0pt #1$}}{}}

%%% Работа с картинками
\usepackage{graphicx}  % Для вставки рисунков
\graphicspath{{images/}{images2/}}  % папки с картинками
\setlength\fboxsep{3pt} % Отступ рамки \fbox{} от рисунка
\setlength\fboxrule{1pt} % Толщина линий рамки \fbox{}
\usepackage{wrapfig} % Обтекание рисунков текстом

%%% Работа с таблицами
\usepackage{array,tabularx,tabulary,booktabs} % Дополнительная работа с таблицами
\usepackage{longtable}  % Длинные таблицы
\usepackage{multirow} % Слияние строк в таблице

%%% Теоремы
\theoremstyle{plain} % Это стиль по умолчанию, его можно не переопределять.
\newtheorem{theorem}{Теорема}[section]
\newtheorem{proposition}[theorem]{Утверждение}
 
\theoremstyle{definition} % "Определение"
\newtheorem{corollary}{Следствие}[theorem]
\newtheorem{problem}{Задача}[section]
 
\theoremstyle{remark} % "Примечание" \newtheorem*{nonum}{Решение}
%%% Программирование
\usepackage{etoolbox} % логические операторы

%%% Страница
\usepackage{extsizes} % Возможность сделать 14-й шрифт
\usepackage{geometry} % Простой способ задавать поля
	\geometry{top=25mm} \geometry{bottom=35mm}
	\geometry{left=35mm}
	\geometry{right=20mm}
 %
\usepackage{fancyhdr} % Колонтитулы
 	\pagestyle{fancy}
 	\renewcommand{\headrulewidth}{0mm}  % Толщина линейки, отчеркивающей верхний колонтитул
 	\rhead{Exam}
 	\chead{Calculus}
 	\lhead{\today}
 	% \cfoot{Нижний в центре} % По умолчанию здесь номер страницы

\usepackage{setspace} % Интерлиньяж
%\onehalfspacing % Интерлиньяж 1.5
%\doublespacing % Интерлиньяж 2
%\singlespacing % Интерлиньяж 1

\usepackage{lastpage} % Узнать, сколько всего страниц в документе.

\usepackage{soul} % Модификаторы начертания

\usepackage{hyperref} \usepackage[usenames,dvipsnames,svgnames,table,rgb]{xcolor}
\hypersetup{				% Гиперссылки
    unicode=true,           % русские буквы в раздела PDF
    pdftitle={Заголовок},   % Заголовок
    pdfauthor={Автор},      % Автор
    pdfsubject={Тема},      % Тема
    pdfcreator={Создатель}, % Создатель
    pdfproducer={Производитель}, % Производитель
    pdfkeywords={keyword1} {key2} {key3}, % Ключевые слова
    colorlinks=true,       	% false: ссылки в рамках; true: цветные ссылки
    linkcolor=red,          % внутренние ссылки
    citecolor=green,        % на библиографию
    filecolor=magenta,      % на файлы
    urlcolor=cyan           % на URL
}

%\renewcommand{\familydefault}{\sfdefault} % Начертание шрифта

\usepackage{multicol} % Несколько колонок

\author{Кутузов Дмитрий}
\title{Ответы на билеты по математическому анализу}
\date{\today}

\begin{document} % конец преамбулы, начало документа

\maketitle

{
	\hypersetup{linkcolor=black}
	\tableofcontents
	\newpage
}


\addcontentsline{toc}{section}{Производные высших порядков, правила их вычисления.}
\section*{Дифференцируемость и непрерывность функций одной переменной.}


\addcontentsline{toc}{subsection}{Производные высших порядков, правила их вычисления.}
\subsection*{Производные высших порядков, правила их вычисления.}

Пусть $f: X \rightarrow \mathbf{R}$, $x_o$ - внутренняя точка области определения $X$, и пусть в некоторой окрестности $U(x_0)$ точки $x_0$ везде существует производная. Тогда в окрестности $U(x_0)$ определена функция $\phi(x) = f'(x)$, поэтому $x_0$ - внутренняя точка области определения функции $\phi$. Значит в этой точке определена производная для функции $\phi$, называемая второй производной функции $f$ (или производной второго порядка) в точке $x_0$:

\[
	f''(x) = \phi'(x_0) \text{, или } f''(x_0) = (f')'(x_0)
\]

Аналогично определяется производная третьего, четвертого порядка и так далее:

\[
	f'''(x_0) = (f'')'(x_0)
\]
\[
	f^{(4)}(x_0) = (f''')'(x_0)
\]
\[
	...
\]
\[
	f^{(n)}(x_0) = (f^{(n-1)})'(x_0)
\]

\textbf{Правила вычисления производных высших порядков}

\[
	(c \cdot f)^{(n)} = c \cdot f^{(n)}
\]

\[
	(f + g)^{(n)} = f^{(n)} + g^{(n)}
\]

\[
	(f \cdot g)^{(n)} = \sum_{k = 0}^{n} C_n^k \cdot f^{(n-k)} \cdot g^{(k)}
\]


\newpage
\addcontentsline{toc}{subsection}{Производная от функции, заданной параметрически.}
\subsection*{Производная от функции, заданной параметрически.}

Говорят, что функция $y(x)$ задана параметрически, если и переменная $x$, и функция $y$ заданы как функции некоторого параметра $t$:

\[
	y(x) =
	\begin{cases}
		x = x(t) \\
		y = y(t)
	\end{cases} t\in T
\]

Однако чаще всего найти явное выражение для $y(x)$ сложно или невозможно. Как считать производную функции, заданной параметрически, когда нельзя выразить функцию явно?

Из формулы вычисления дифференциала \[ dy = y'(x) \cdot dx \] учитывая равенства $ dx = x'(t) \cdot dt $ и $ dy = y'(t) \cdot dt $ получаем \[ y'(x) = \frac{dy}{dx} = \frac{y'(t) \cdot dt}{x'(t) \cdot dt} = \frac{y'(t)}{x'(t)} \]

Вторая производная $y''(x) = (y'(x))'$:
\[
	y''(x) = \frac{(y'(x))'_t}{x'_t} = \frac{1}{x'_t} \cdot \left( \frac{y'(t)}{x'(t)} \right)'_t
\]

или

\[
	\frac{d^2y}{dx^2} = y''_{xx} = (y'_x)' = \frac{dy'_x}{dx} = \frac{(y'_x)'_t}{x'_t}
\]

\newpage
\addcontentsline{toc}{subsection}{Производная от функции, заданной неявно. }
\subsection*{Производная от функции, заданной неявно.                                                   }

Неявное задание - один из способов задания функциию Функция задана явно, если она задана уравнением $y = y(x)$. Уравнение $F(x, y) = 0$ задает функцию $y(x)$ неявно. Одну и ту же функцию можно задать разными способами (например уравнение окружности)

\textbf{Как считать производную функции заданной неявно}

Нужно равенство $F(x, y) = 0$ дифференцировать как тождество, считая $x$ независимой переменной, а $y$ - функцией от $x$



\newpage
\addcontentsline{toc}{subsection}{Дифференциалы высших порядков и их свойства. }
\subsection*{Дифференциалы высших порядков и их свойства.                                               }
\textit{\textbf{Определение:}} Дифференциал второго порядка - это дифференциал от дифференциала первого порядка:
\[
	d^2y = d(dy)
\]

\textbf{\textit{Выведем формулу для вычисления дифференциала второго порядка.}}

Дифференциал первого порядка вычисляется по формуле
\[
	dy = y'(x) dx
\]

Дифференциал первого порядка - это функция от двух переменных: $x$ и приращения $dx$.

Зафиксируем $dx$, будем считать, что меняется только переменная $x$.

Подставляя в формулу $d^2y = d(dy)$ для второго дифференциала формулу для вычисления первого $dy = y'(x)dx$ и пользуясь правилами вычисления дифференциала, получаем
\[
	d^2y = d(y'(x) \cdot dx) = (y'(x)dx)'dx = y''(x)(dx)^2
\]

Итак, получили формулу для вычисления дифференциала второго порядка:

\[
	d^2y = y''(x) \cdot dx^2
\]

Аналогично выводится формула для вычисления дифференциала $n$-го порядка:
\[
	d^ny = y^{(n)}dx^n
\]

\textbf{Свойства дифференциала порядка $n$:}

\[
	d^n(c\cdot f) = c\cdot d^n f
\]
\[
	d^n(f + g) = d^nf + d^ng
\]
\[
	d^n(f\cdot g) = \sum_{k = 0}^n C_n^k \cdot d^{n-k}f \cdot d^k g
\]

Свойство инвариантности, справедливое для дифференциала первого порядка, для дифференциала $n$ порядка в общем случае не выполняется. Действительно, пусть переменная $x$ является функцией от новой переменной $x = x(t)$.

Тогда пользуясь правилом для вычисления дифференциала от произведения, получим:
\[
	d^2y = d(y'(x)dx) = d(y'(x)) \cdot dx + y'(x) \cdot d^2x
\]
\[
	d^2y = y''(x) \cdot dx^2 + y'(x) \cdot x''_t dt^2
\]

Формулы (1) и (2) отличаются вторым слагаемым. Если оно не равно нулю, то свойство инвариантности дифференциала 2-го порядка не выполняется. Второе слагаемое обращается в нуль в том случае, если функция $x(t)$ линейна. Поэтому при линейной замене $x(t) = at + b$ свойство инвариантности дифференциала 2-го (и n-го) порядка верно.




\newpage
\addcontentsline{toc}{subsection}{Теорема Ферма. }
\subsection*{Теорема Ферма.                                                                             }

Пусть выполняются следующие условия:

\begin{enumerate}
	\item Функция определена на промежутке $X$
	\item $x_0$ - внутренняя точка промежутка $X$
	\item функция в точке $x_0$ принимает наибольшее значение, т.е. $f(x) \leq f(x_0)$ для всех точек из промежутка $X$
	\item Существует конечная производная в точке $x_0$
\end{enumerate}
\[ \textbf{Тогда } f'(x_0) = 0 \]

\textit{\textbf{Доказательство:}}

\[
	f'(x_0) = \lim_{\Delta x \rightarrow 0} \frac{f(x_0 + \Delta x) - f(x_0)}{\Delta x} = A
\]

По теории о связи существования предела и односторонних пределов:

\[
	\exists f'(x_0) = A \in \mathbf(R) \Leftrightarrow \exists f'_-(x_0) = f'_+(x_0) = A
\]

\[
	f'_-(x_0) = f'_+(x_0) = \lim_{\Delta x \rightarrow 0 + 0} \frac{f(x_0 + \Delta x) - f(x_0)}{\Delta x} = f'(x_0) = 0
\]
Что и требовалось доказать


\newpage
\addcontentsline{toc}{subsection}{Теорема Ролля. }
\subsection*{Теорема Ролля.                                                                             }

\begin{enumerate}
	\item Функция $f$ определена и непрерывна на отрезке $[a, b]$
	\item Функция $f$ дифференцируема на $(a, b)$
	\item $f(a) = f(b)$
\end{enumerate}

Тогда внутри $(a, b)$ найдется точка, в которой производная функции равна нулю:
\[
	x_0 \in (a, b): f'(x_0) = 0
\]

\textit{\textbf{Доказательство:}}

\textbf{Вторая теорема Вейерштрасса:}

Если функция определена и непрерывна на отрезке, то она достигает на нем своих точных верхней и нижней граней.

Обозначим
\[ M = sup_{x \in [a, b]}f(x); m = inf_{x \in [a, b]} f(x) \]

\begin{enumerate}
	\item $m = M: \forall x_0 \in (a, b): f(x_0) = f(a) = f(b) \Rightarrow f(x) = const \Rightarrow f'(x_0) = 0$
	\item $m < M$
\end{enumerate}

Поскольку функция на концах отрезка принимает одинаковые значения, то одно из значений (либо $m$, либо $M$) достигается во внутренней точке $x_0$.

Тогда для точки $x_0$ выполняются все условия теоремы Ферма, поэтому $f'(x_0) = 0$




\newpage
\addcontentsline{toc}{subsection}{Теорема Лагранжа. }
\subsection*{Теорема Лагранжа.                                                                          }

Пусть
\begin{enumerate}
	\item Функция $f$ определена и непрерывна на отрезке $[a, b]$
	\item Функция $f$ дифференцируема на $(a, b)$
\end{enumerate}
Тогда
\[
	\exists c \in (a, b): f'(c) = \frac{f(b) - f(a)}{b - a}
\]

\textit{\textbf{Доказательство:}}

Введем вспомогательную функцию $F(x) = f(x) - \lambda x$
\[
	F(a) = F(b): f(a) - \lambda a = f(b) - \lambda b
\]

\[
	\lambda = \frac{f(b) - f(a)}{b - a}
\]

То есть \[F(x) = f(x) - \frac{f(b) - f(a)}{b - a}\]

Видим, что функция $F(x)$ удовлетворяет условиям теоремы Ролля: непрерывна на $[a, b]$ как разность двух непрерывных функций, дифференцируема на интегрвале как разность двух дифференцируемых функций, и $F(a) = F(b)$. Значит по теореме Ролля $\exists c \in (a, b): F'(c) = 0$.

\[ F'(x) = f'(x) - \frac{f(b) - f(a)}{b - a}, f'(c) - \frac{f(b) - f(a)}{b - a} = 0\]

Отсюда

\[ f'(c) = \frac{f(b) - f(a)}{b - a} \]
Ч.т.д



\newpage
\addcontentsline{toc}{subsection}{Теорема Коши. }
\subsection*{Теорема Коши.                                                                              }
Пусть
\begin{enumerate}
	\item Функция $f$ и $g$ определены и непрерывны на отрезке $[a, b]$
	\item $\forall x \in (a, b) \exists f'(x) \in \mathbf{R}, g'(x) \in \mathbf{R}, g'(x) \neq 0 $
\end{enumerate}

Тогда существует точка $c \in (a, b)$, такая что

\[
	\frac{f'(c)}{g'(c)} = \frac{f(b) - f(a)}{g(b) - g(a)}
\]

\textit{\textbf{Доказательство:}}

1) Пусть $g(b) = g(a): \exists x_0 \in (a, b): g'(x_0) = 0$ (по теореме Ролля) $\Rightarrow$ противоречие с условием

2) $g(b) \neq g(a)$.

Введем вспомогательную функцию $F(x) = f(x) - \lambda g(x)$

\[
	F(a) = F(b): f(a) - \lambda g(a) = f(b) - \lambda g(b) \Rightarrow \lambda  = \frac{f(b) - f(a)}{g(b) - g(a)}
\]

\[
	F(x) = f(x) - \frac{f(b) - f(a)}{g(b) - g(a)}g(x)
\]

$F(x)$ - удовлетворяет условия теореме Ролля (непрерывна на отрезке [a, b] дифференцируема на (a, b), F(a) = F(b))

Тогда $\exists c \in (a, b): F'(c) = 0$, $F'(c) = f'(c) - \lambda g'(c) = 0$

\[
	\lambda = \frac{f'(c)}{g'(c)} = \frac{f(b) - f(a)}{g(b) - g(a)}
\]

Ч.т.д


\newpage
\addcontentsline{toc}{subsection}{Правила Лопиталя. }
\subsection*{Правила Лопиталя.                                                                          }

\textbf{\textit{Теорема 1.} Правило Лопиталя для $\frac{0}{0}$ в случае конечного промежутка.}

Пусть:
\begin{equation*}
	\begin{aligned}
		 & \text{(1) } f(x) \text{ и } g(x) \text{ определены и дифференцируемы в промежутке }(a, b)    \\
		 & \text{(2) } \forall x \in (a, b): \; g'(x) \neq 0                                            \\
		 & \text{(3) }  \lim_{x-> a + 0} f(x) =  \lim_{x-> a + 0} g(x) = 0                              \\
		 & \text{(4) }  \exists \lim_{x-> a + 0} \dfrac{f'(x)}{g'(x)} \text{ конечный или бесконечный } \\
	\end{aligned}
\end{equation*}

\begin{center}
	$\Downarrow$
\end{center}

\[\lim_{x-> a + 0} \dfrac{f(x)}{g(x)} =\lim_{x-> a + 0} \dfrac{f'(x)}{g'(x)} \]



\textbf{\textit{Теорема 2.} Правило Лопиталя для $\frac{0}{0}$ в случае бесконечного промежутка.}

Пусть:
\begin{equation*}
	\begin{aligned}
		 & \text{(1) } f(x) \text{ и } g(x) \text{ определены на луче }[a, +\infty)                                        \\
		 & \text{(2) } f \text{ и } g \text{ дифференцируемы на } (a, +\infty), g'(x) \neq 0 \; \forall x \in (a, +\infty) \\
		 & \text{(3) }  \lim_{x-> +\infty} f(x) =  \lim_{x-> +\infty} g(x) = 0                                             \\
		 & \text{(4) }  \exists \lim_{x-> +\infty} \dfrac{f'(x)}{g'(x)} \text{ конечный или бесконечный }                  \\
	\end{aligned}
\end{equation*}

\begin{center}
	$\Downarrow$
\end{center}

\[\lim_{x-> +\infty} \dfrac{f(x)}{g(x)} =\lim_{x->+\infty } \dfrac{f'(x)}{g'(x)} \]

\textbf{\textit{Теорема 3.} Правило Лопиталя для $\frac{\infty}{\infty}$ в случае конечного промежутка.}

Пусть:
\begin{equation*}
	\begin{aligned}
		 & \text{(1) } f(x) \text{ и } g(x) \text{ определены на интервале }(a, b)                                 \\
		 & \text{(2) } f \text{ и } g \text{ дифференцируемы на } (a, b)  \; \forall x \in (a, b), \; g'(x) \neq 0 \\
		 & \text{(3) }  \lim_{x-> a + 0} f(x) =  \lim_{x-> a + 0} g(x) = \infty                                    \\
		 & \text{(4) }  \exists \lim_{x-> a + 0 } \dfrac{f'(x)}{g'(x)} \text{ конечный или бесконечный }           \\
	\end{aligned}
\end{equation*}

\begin{center}
	$\Downarrow$
\end{center}

\[\lim_{x-> a + 0} \dfrac{f(x)}{g(x)} =\lim_{x-> a + 0 } \dfrac{f'(x)}{g'(x)} \]



\textbf{\textit{Теорема 4.} Правило Лопиталя для $\frac{\infty}{\infty}$ в случае бесконечного промежутка.}

Пусть:
\begin{equation*}
	\begin{aligned}
		 & \text{(1) } f(x) \text{ и } g(x) \text{ определены на луче }[a, +\infty)                                            \\
		 & \text{(2) } f \text{ и } g \text{ дифференцируемы на } (a, +\infty)  \; \forall x \in (a, +\infty), \; g'(x) \neq 0 \\
		 & \text{(3) }  \lim_{x-> + \infty} f(x) =  \lim_{x-> +\infty} g(x) = \infty                                           \\
		 & \text{(4) }  \exists \lim_{x-> +\infty } \dfrac{f'(x)}{g'(x)} \text{ конечный или бесконечный }                     \\
	\end{aligned}
\end{equation*}

\begin{center}
	$\Downarrow$
\end{center}

\[\lim_{x-> + \infty} \dfrac{f(x)}{g(x)} =\lim_{x-> + \infty } \dfrac{f'(x)}{g'(x)} \]

\textbf{\textit{Теорема 1 + 2.} Общая для $\frac{0}{0}$}

\[
	\lim_{x -> a} \frac{f(x)}{g(x)} = \frac{0}{0} = \lim_{x->a} \frac{f'(x)}{g'(x)}, a \in \overline{ \mathbb{R}}
\]



\textbf{\textit{Теорема 3 + 4.} Общая для $\frac{\infty}{\infty}$}

\[
	\lim_{x -> a} \frac{f(x)}{g(x)} = \frac{\infty}{\infty} = \lim_{x->a} \frac{f'(x)}{g'(x)}, a \in \overline{ \mathbb{R}}
\]



\newpage

\addcontentsline{toc}{subsection}{Формула Тейлора с различными формами остаточного члена. }
\subsection*{Формула Тейлора с различными формами остаточного члена.                                   }

\begin{equation*}
	\begin{aligned}
		 & T_n = \sum_{k=0}^n b_k \cdot (x - x_0)^k \\
		 & b_k = \frac{f^{(k)}(x_0)}{k!}
	\end{aligned}
\end{equation*}

\textbf{\textit{Определение:}}
Пусть функция $f(x)$ n-жды дифференцируемая функция.
Тогда многочлен:

\[
	T_n = f(x_0) + f'(x_0)(x-x_0) + \frac{f''(x_0)}{2!}(x-x_0)^2 + \dots + \frac{f^{(n)}(x_0)}{n!}(x-x_0)^n = \] \[ = \sum_{k=0}^n \frac{f^{(k)}(x_0)}{k!} \cdot (x - x_0)^k
\]

Называется многочленом Тейлора для функции $f(x)$ в точке $x_0$


\begin{center}
	\textbf{Основное свойство многочлена Тейлора}
\end{center}

В точке $x_0$ совпадают значения функции и её многочлена Тейлора, а также значения их первых $n$ производных.

\[
	T_n(x_0) = f(x_0), T_n'(x_0) = f'(x_0) \dots T_n^{(n)}(x_0) = f^{(n)}(x_0)
\]

\begin{center}
	\textbf{Аналитическая функция}
\end{center}

\textbf{Аналитическая функция вещественной переменной} — функция, которая совпадает со своим рядом Тейлора в окрестности любой точки области определения.

\begin{center}
	\textbf{Формула Тейлора произвольной функции}
\end{center}

\[ f(x) = T_n(x) + r_n(x) \]
Где $r_n(x)$ - остаточный член формулы Тейлора

\begin{center}
	\textbf{Форма Пеано}
\end{center}

\[ r_n(x) = o((x-x_0)^n) \]

\begin{center}
	\textbf{Форма Лагранжа}
\end{center}

\[ r_n = \frac{f^{(n+1)}(c)}{(n+1)!}(x-x_0)^{n+1} \]

\begin{center}
	\textbf{Формула Маклорена}
\end{center}

Частный случай ряда Тейлора при $x_0 = 0$
\[ f(x) = f(0) + f'(0) + \frac{f''(0)}{2!}x^2 + \dots + \frac{f^{(n)}(0)}{n!}x^n + o(x^n) \]



\newpage

\addcontentsline{toc}{subsection}{Стандартные разложения по формуле Маклорена. }
\subsection*{Стандартные разложения по формуле Маклорена.                                              }

\begin{center}
	\textbf{Разложения некоторых элементарных функций по Маклорену}
\end{center}

Все $r_n$ в форме Лагранжа.

\begin{equation*}
	\begin{aligned}
		\hline
		 & e^x = 1 + x + \frac{x^2}{2!} + \frac{x^3}{3!} + \dots + \frac{x^n}{n!} + o(x^n) \text{ при }x\rightarrow0                      \\
		 & r_n  = \frac{e^c}{(n+1)!}x^{n+1}                                                                                               \\
		\\ \hline \\
		 & ch(x) = \frac{e^x - e^{-1}}{2} = 1 + \frac{x^2}{2!} + \dots + \frac{x^{2n}}{(2n)!} + o(x^{2n+1})                               \\
		\\ \hline \\
		 & sh(x) = x + \frac{x^3}{3!} + \frac{x^5}{5!} + \dots + \frac{x^{2n+1}}{(2n+1)!} + \dots                                         \\
		\\ \hline \\
		 & sin(x) = x - \frac{x^3}{3!} + \frac{x^5}{5!} - \dots + (-1)^k \frac{x^{2n+1}}{(2n+1)!} + o(x^{2n+2})                           \\
		\\ \hline \\
		 & cos(x) = 1 - \frac{x^2}{2!} + \frac{x^4}{4!} - \frac{x^6}{6!} + \dots + (-1)^{n}\frac{x^{2n}}{(2n)!}                           \\
		\\ \hline \\
		 & ln(1+x) = x - \frac{x^2}{2} + \frac{x^3}{3} - \dots + (-1)^{n+1}\frac{x^n}{n}                                                  \\
		\\ \hline \\
		 & (1 + x)^\alpha = 1 + \alpha x + \frac{\alpha (\alpha-1)}{2!}x^2 + \dots + \frac{\alpha(\alpha-1)\dots(\alpha-n+1)}{n!} + \dots
		\\ \hline
	\end{aligned}
\end{equation*}



\newpage
\addcontentsline{toc}{subsection}{Условия постоянства функции на промежутке.}
\subsection*{Условия постоянства функции на промежутке.                                                }

\textbf{\textit{Теорема.}} Пусть функция $f(x)$ определена и непрерывна на некотором промежутке $X$, и в каждой внутренней точке этого промежутка существует конечная производная. Тогда функция $f(x)$ постоянна на $X$ тогда и только тогда, когда $f'(x) = 0$ в любой внутренней точке промежутка $X$.

(то есть $\forall x \in X \; f(x) = C \Leftrightarrow \forall x \in X f'(x) = 0$, где $X$ - множество внутренних точек $X$)


\newpage
\addcontentsline{toc}{subsection}{Определение функции, монотонной на промежутке. Критерии строгой и нестрогой монотонности. }
\subsection*{Определение функции, монотонной на промежутке. Критерии строгой и нестрогой монотонности. }

Если функция возрастает или убывает на некотором промежутке, то она называется монотонной на этом промежутке.

\textit{\textbf{Теорема 1. (Критерий нестрогой монотонности.)}}

Пусть функция $f(x)$ определена и непрерывна на промежутке $X$, дифференцируема в любой внутренней точке этого промежутка.

Тогда
\[ f(x) \text{ - неубывающая на }X \Leftrightarrow \forall x \in X \; f'(x) \geq 0 \]

\textit{\textbf{Теорема 2. (Критерий строго монотонности.)}}
Пусть функция $f$ определена и непрерывна на промежутке $X$, дифференцируема в любой внутренней точке этого промежутка.

Тогда
\[
	f(x) \text{ строго возрастает } \Leftrightarrow \]

$\text{1) } \forall x \in X \; f'(x) \geq 0  $

$\text{2) множество решений уравнения } f'(x) = 0 \text{ не содержит интервалов} $


\newpage
\addcontentsline{toc}{subsection}{Определение точки локального экстремума. Необходимое условие экстремума. Достаточные условия экстремума. }
\subsection*{Определение точки локального экстремума. Необходимое условие экстремума. Достаточные условия экстремума. }

\textit{\textbf{Определение.}} Пусть $f: X \rightarrow \mathbf{R}$. Точка $x_0 \in X$ называется точкой (нестрогого) локального максимума функции $f$, если существует окрестность $U(x_0)$ точки $x_0$, такая что $\forall x \in U(x_o)$ $f(x) \leq f(x_0)$. Соответственно, точка $x_0 \in X$ называется точкой (нестрогого) локального минимума функции $f$, если $\exists U(x_0): \forall x \in U(x_0) \; f(x) \geq f(x_0)$

Значение $f(x_0)$ функции $f$ в точке локального максимум (или минимума) называется локальным максимумом (или минимумом).

Точки локального максимума или минимума называются точками локального экстремума. Любой максимум или минимум называется экстремумом.

\subsubsection*{Теорема (необходимое условие экстремума)}

Пусть функция $f$ определена и непрерывна в окрестности точки $x_0$. Тогда если $x_0$ - точка максимума функции $f$, то $f'(x_0)$ либо равна нулю, либо не существует конечная.

\subsubsection*{Теорема (Первое достаточное условие экстремума)}

Пусть существует окрестность $U(x_0)$ точки $x_0$, такая что:

\begin{enumerate}
	\item функция $f$ определена и непрерывна в $U(x_0)$
	\item дифференцируема в $U(x_0)$, кроме, может быть, самой точки $x_0$
\end{enumerate}
Тогда:
\begin{enumerate}
	\item если производная в точке $x_0$ меняет знак с + на -
	\item если производная в точке $x_0$ меняет знак с - на +
\end{enumerate}

\subsubsection*{Доказательство:}

По критерию нестрогой монотонности функции:

\[
	\forall x \in (x_0 - \delta, x_0) \;  f'(x_0) > 0
\]
то $f(x)$ - неубывающая на $(x_0 - \delta, x_0)$  Поэтому $\forall x \in (x_0 - \delta, x_0)$ выполняется неравенство $f(x) \leq f(x_0)$

Соответственно \[\forall x \in (x_0, x_0 + \delta) \; f'(x_0) \leq 0 \] то $f(x)$ - невозрастающая на $(x_0, x_0 + \delta)$. Поэтому $\forall x \in (x_0, x_0 + \delta)$ выполняется неравенств $f(x_0) \geq f(x_0)$
Следовательно, $\forall x \in U(x_0) \; f(x) \leq f(x_0)$
По определению это значит, что $x_0$ - нестрогий локальных максимум.

\subsubsection*{Теорема (Второе достаточное условие)}

Пусть функция $f$ определена и дважды дифференцируема в некоторой окрестности $U(x_0)$, $f'(x_0) = 0$ (т.е. $x_0$ - стационарная точка)

Тогда:
\begin{enumerate}
	\item Если $f''(x_0) > 0$, то $x_0$ - локальный минимум.
	\item Если $f''(x_0) < 0$, то $x_0$ - локальный максимум.
\end{enumerate}

\subsubsection*{Доказательство:}

1) $f''(x_0) > 0$. Так как по условию $f'(x_0) = 0$, то
\[ c = f''(x_0) = \lim_{\Delta x \rightarrow 0} \frac{f'(\Delta x + x_0) - f'(x_0)}{\Delta x} = \lim_{\Delta x \rightarrow 0} \frac{f'(\Delta x + x_0)}{\Delta x} > 0 \]

По определению предела для $\forall \epsilon = \frac{c}{2} > 0$ найдется $\delta > 0$, такое что
\[ \forall \Delta x: 0 < |\Delta x| < \delta : \]
\[ \Big| \frac{f'(\Delta x + x_0)}{\Delta x} - c \Big| < \epsilon = \frac{c}{2} \]

\[
	\begin{cases}
		 & \frac{f'(\Delta x + x_0)}{\Delta x} - c < \epsilon = \frac{c}{2} \\
		 & \frac{f'(\Delta x + x_0)}{\Delta x} - c > - \frac{c}{2}          \\
	\end{cases}
\]

Значит $\frac{f'(\Delta x + x_0)}{\Delta x} > \frac{c}{2} > 0$ при $0 < |\Delta x| < \delta$ числитель и знаменатель должны иметь одинаковые знаки.

Если $\Delta x > 0$, то $f'(\Delta x + x_0) > 0$

Если $\Delta x < 0$ слева от $x_0$, то $f'(\Delta x + x_0) < 0$

Значит по 1 достаточному условию, $x_0$ - локальный минимум функции $f$, так как в этой точке производная меняет знак с - на +.


\newpage
\addcontentsline{toc}{subsection}{Определение выпуклости функции на промежутке. Критерии выпуклости. }
\subsection*{Определение выпуклости функции на промежутке. Критерии выпуклости.                                       }
Пусть функция $f$ определена и на промежутке $X$, а также дифференцируема в любой его внутренней точке.

\subsubsection*{Определение 1.}
Функция, непрерывная на промежутке, и дифференцируемая в любой его внутренней точке, называется выпуклой вниз на этом промежутке, если в любой его внутренней точке касательная к графику лежит ниже самого графика на этом промежутке.

То есть $\forall x_0 \in X, \; \forall x \in X$ выполняется неравенство $f(x_0) + f'(x_0) \cdot (x - x_0) \leq f(x)$

\subsubsection*{Определение 2.}
Функция, непрерывная на промежутке, и дифференцируемая в любой его внутренней точке, называется выпуклой вверх на этом промежутке, если в любой его внутренней точке касательная к графику лежит выше самого графика на этом промежутке.

То есть $\forall x_0 \in X, \; \forall x \in X$ выполняется неравенство $f(x_0) + f'(x_0) \cdot (x - x_0) \geq f(x)$

\subsubsection*{Теорема. (Первый критерий выпуклости функции на промежутке.)}
Пусть функция $f$ определена и непрерывна на промежутке $X$, а также дифференцируема в люой его внутренней точке. Тогда:

\begin{enumerate}
	\item $f$ выпукла вниз на $X \Leftrightarrow f'$ - неубывающая функция на $X$
	\item $f$ выпукла вверх на $X \Leftrightarrow f'$ - невозрастающая функция на $X$
\end{enumerate}

\subsubsection*{Теорема. (Второй критерий выпуклости функции на промежутке.)}
Пусть функция $f$ определена и непрерывна на промежутке $X$, а также дважды дифференцируема в любой внутренней точке этого промежутка.
Тогда:

\begin{enumerate}
	\item $f$ выпукла вних на $X \Leftrightarrow \forall x \in X \;  f'(x) \geq 0$
	\item $f$ выпукла вверх на $X \Leftrightarrow \forall x \in X \; f'(x) \leq 0$
\end{enumerate}








\newpage
\addcontentsline{toc}{subsection}{Точки перегиба. Необходимые и достаточные условия точки перегиба.   }
\subsection*{Точки перегиба. Необходимые и достаточные условия точки перегиба.                                        }

\subsubsection*{Определение.}
Точка $x_0$ - называется точкой перегиба функции, если функция определена и непрерывна в некоторой окрестности точки $x_0$, в левой части окрестности выпукла в одну сторону, а в правой в другую.

\subsubsection*{Теорема (Необходимое условие точки перегиба).}
Если $x_0$ - точка перегиба функции, то $f'(x_0)$ либо равно нулю, либо не существует конечная.

\subsubsection*{Теорема. (Первое достаточное условие перегиба).}
Пусть в некоторой проколотой окрестности точки $x_0$ существует вторая производная. Если в левой части этой окрестности вторая производная имеет один знак, а в правой - другой, то $x_0$ - точка перегиба.

\subsubsection*{Теорема. (Второе достаточное условие перегиба).}
Пусть функция $f$ дважды дифференцируема в некотрой окрестности точки $x_0$, $f'(x_0) = 0$, и $\exists f''(x_0) \neq 0$. Тогда $x_0$ - точка перегиба функции $f$.

\newpage
\addcontentsline{toc}{subsection}{Асимптоты вертикальные, наклонные и горизонтальные.    }
\subsection*{Асимптоты вертикальные, наклонные и горизонтальные.                                                      }


\addcontentsline{toc}{subsection}{Схема исследования функции и построения ее графика.    }
\subsection*{Схема исследования функции и построения ее графика.                                                      }



\end{document} % конец документа

% Этот шаблон документа разработан в 2014 году
% Данилом Фёдоровых (danil@fedorovykh.ru) 
% для использования в курсе 
% <<Документы и презентации в \LaTeX>>, записанном НИУ ВШЭ
% для Coursera.org: http://coursera.org/course/latex .
% Исходная версия шаблона --- 
% https://www.writelatex.com/coursera/latex/3.2

\documentclass[a4paper,12pt]{article}

%%% Работа с русским языком
\usepackage{cmap}					% поиск в PDF
\usepackage{mathtext} 				% русские буквы в фомулах
\usepackage[T2A]{fontenc}			% кодировка
\usepackage[utf8]{inputenc}			% кодировка исходного текста
\usepackage[english,russian]{babel}	% локализация и переносы

%%% Дополнительная работа с математикой
\usepackage{amsmath,amsfonts,amssymb,amsthm,mathtools} % AMS
\usepackage{icomma} % "Умная" запятая: $0,2$ --- число, $0, 2$ --- перечисление

%% Номера формул
%\mathtoolsset{showonlyrefs=true} % Показывать номера только у тех формул, на которые есть \eqref{} в тексте.
%\usepackage{leqno} % Немуреация формул слева

%% Свои команды
\DeclareMathOperator{\sgn}{\mathop{sgn}}

%% Перенос знаков в формулах (по Львовскому)
\newcommand*{\hm}[1]{#1\nobreak\discretionary{}
{\hbox{$\mathsurround=0pt #1$}}{}}

%%% Работа с картинками
\usepackage{graphicx}  % Для вставки рисунков
\graphicspath{{images/}{images2/}}  % папки с картинками
\setlength\fboxsep{3pt} % Отступ рамки \fbox{} от рисунка
\setlength\fboxrule{1pt} % Толщина линий рамки \fbox{}
\usepackage{wrapfig} % Обтекание рисунков текстом

%%% Работа с таблицами
\usepackage{array,tabularx,tabulary,booktabs} % Дополнительная работа с таблицами
\usepackage{longtable}  % Длинные таблицы
\usepackage{multirow} % Слияние строк в таблице

%%% Теоремы
\theoremstyle{plain} % Это стиль по умолчанию, его можно не переопределять.
\newtheorem{theorem}{Теорема}[section]
\newtheorem{proposition}[theorem]{Утверждение}
 
\theoremstyle{definition} % "Определение"
\newtheorem{corollary}{Следствие}[theorem]
\newtheorem{problem}{Задача}[section]
 
\theoremstyle{remark} % "Примечание" \newtheorem*{nonum}{Решение}
%%% Программирование
\usepackage{etoolbox} % логические операторы

%%% Страница
\usepackage{extsizes} % Возможность сделать 14-й шрифт
\usepackage{geometry} % Простой способ задавать поля
	\geometry{top=25mm} \geometry{bottom=35mm}
	\geometry{left=35mm}
	\geometry{right=20mm}
 %
\usepackage{fancyhdr} % Колонтитулы
 	\pagestyle{fancy}
 	\renewcommand{\headrulewidth}{0mm}  % Толщина линейки, отчеркивающей верхний колонтитул
 	\rhead{Exam}
 	\chead{Calculus}
 	\lhead{\today}
 	% \cfoot{Нижний в центре} % По умолчанию здесь номер страницы

\usepackage{setspace} % Интерлиньяж
%\onehalfspacing % Интерлиньяж 1.5
%\doublespacing % Интерлиньяж 2
%\singlespacing % Интерлиньяж 1

\usepackage{lastpage} % Узнать, сколько всего страниц в документе.

\usepackage{soul} % Модификаторы начертания

\usepackage{hyperref} \usepackage[usenames,dvipsnames,svgnames,table,rgb]{xcolor}
\hypersetup{				% Гиперссылки
    unicode=true,           % русские буквы в раздела PDF
    pdftitle={Заголовок},   % Заголовок
    pdfauthor={Автор},      % Автор
    pdfsubject={Тема},      % Тема
    pdfcreator={Создатель}, % Создатель
    pdfproducer={Производитель}, % Производитель
    pdfkeywords={keyword1} {key2} {key3}, % Ключевые слова
    colorlinks=true,       	% false: ссылки в рамках; true: цветные ссылки
    linkcolor=red,          % внутренние ссылки
    citecolor=green,        % на библиографию
    filecolor=magenta,      % на файлы
    urlcolor=cyan           % на URL
}

%\renewcommand{\familydefault}{\sfdefault} % Начертание шрифта

\usepackage{multicol} % Несколько колонок

\author{Кутузов Дмитрий}
\title{Ответы на билеты по математическому анализу}
\date{\today}

\begin{document} % конец преамбулы, начало документа

\maketitle

{
	\hypersetup{linkcolor=black}
	\tableofcontents
	\newpage
}


\addcontentsline{toc}{section}{Производные высших порядков, правила их вычисления.}
\section*{Дифференцируемость и непрерывность функций одной переменной.}


\addcontentsline{toc}{subsection}{Производные высших порядков, правила их вычисления.}
\subsection*{Производные высших порядков, правила их вычисления.}

Пусть $f: X \rightarrow \mathbf{R}$, $x_o$ - внутренняя точка области определения $X$, и пусть в некоторой окрестности $U(x_0)$ точки $x_0$ везде существует производная. Тогда в окрестности $U(x_0)$ определена функция $\phi(x) = f'(x)$, поэтому $x_0$ - внутренняя точка области определения функции $\phi$. Значит в этой точке определена производная для функции $\phi$, называемая второй производной функции $f$ (или производной второго порядка) в точке $x_0$:

\[
	f''(x) = \phi'(x_0) \text{, или } f''(x_0) = (f')'(x_0)
\]

Аналогично определяется производная третьего, четвертого порядка и так далее:

\[
	f'''(x_0) = (f'')'(x_0)
\]
\[
	f^{(4)}(x_0) = (f''')'(x_0)
\]
\[
	...
\]
\[
	f^{(n)}(x_0) = (f^{(n-1)})'(x_0)
\]

\textbf{Правила вычисления производных высших порядков}

\[
	(c \cdot f)^{(n)} = c \cdot f^{(n)}
\]

\[
	(f + g)^{(n)} = f^{(n)} + g^{(n)}
\]

\[
	(f \cdot g)^{(n)} = \sum_{k = 0}^{n} C_n^k \cdot f^{(n-k)} \cdot g^{(k)}
\]


\newpage
\addcontentsline{toc}{subsection}{Производная от функции, заданной параметрически.}
\subsection*{Производная от функции, заданной параметрически.}

Говорят, что функция $y(x)$ задана параметрически, если и переменная $x$, и функция $y$ заданы как функции некоторого параметра $t$:

\[
	y(x) =
	\begin{cases}
		x = x(t) \\
		y = y(t)
	\end{cases} t\in T
\]

Однако чаще всего найти явное выражение для $y(x)$ сложно или невозможно. Как считать производную функции, заданной параметрически, когда нельзя выразить функцию явно?

Из формулы вычисления дифференциала \[ dy = y'(x) \cdot dx \] учитывая равенства $ dx = x'(t) \cdot dt $ и $ dy = y'(t) \cdot dt $ получаем \[ y'(x) = \frac{dy}{dx} = \frac{y'(t) \cdot dt}{x'(t) \cdot dt} = \frac{y'(t)}{x'(t)} \]

Вторая производная $y''(x) = (y'(x))'$:
\[
	y''(x) = \frac{(y'(x))'_t}{x'_t} = \frac{1}{x'_t} \cdot \left( \frac{y'(t)}{x'(t)} \right)'_t
\]

или

\[
	\frac{d^2y}{dx^2} = y''_{xx} = (y'_x)' = \frac{dy'_x}{dx} = \frac{(y'_x)'_t}{x'_t}
\]

\newpage
\addcontentsline{toc}{subsection}{Производная от функции, заданной неявно. }
\subsection*{Производная от функции, заданной неявно.                                                   }

Неявное задание - один из способов задания функциию Функция задана явно, если она задана уравнением $y = y(x)$. Уравнение $F(x, y) = 0$ задает функцию $y(x)$ неявно. Одну и ту же функцию можно задать разными способами (например уравнение окружности)

\textbf{Как считать производную функции заданной неявно}

Нужно равенство $F(x, y) = 0$ дифференцировать как тождество, считая $x$ независимой переменной, а $y$ - функцией от $x$



\newpage
\addcontentsline{toc}{subsection}{Дифференциалы высших порядков и их свойства. }
\subsection*{Дифференциалы высших порядков и их свойства.                                               }
\textit{\textbf{Определение:}} Дифференциал второго порядка - это дифференциал от дифференциала первого порядка:
\[
	d^2y = d(dy)
\]

\textbf{\textit{Выведем формулу для вычисления дифференциала второго порядка.}}

Дифференциал первого порядка вычисляется по формуле
\[
	dy = y'(x) dx
\]

Дифференциал первого порядка - это функция от двух переменных: $x$ и приращения $dx$.

Зафиксируем $dx$, будем считать, что меняется только переменная $x$.

Подставляя в формулу $d^2y = d(dy)$ для второго дифференциала формулу для вычисления первого $dy = y'(x)dx$ и пользуясь правилами вычисления дифференциала, получаем
\[
	d^2y = d(y'(x) \cdot dx) = (y'(x)dx)'dx = y''(x)(dx)^2
\]

Итак, получили формулу для вычисления дифференциала второго порядка:

\[
	d^2y = y''(x) \cdot dx^2
\]

Аналогично выводится формула для вычисления дифференциала $n$-го порядка:
\[
	d^ny = y^{(n)}dx^n
\]

\textbf{Свойства дифференциала порядка $n$:}

\[
	d^n(c\cdot f) = c\cdot d^n f
\]
\[
	d^n(f + g) = d^nf + d^ng
\]
\[
	d^n(f\cdot g) = \sum_{k = 0}^n C_n^k \cdot d^{n-k}f \cdot d^k g
\]

Свойство инвариантности, справедливое для дифференциала первого порядка, для дифференциала $n$ порядка в общем случае не выполняется. Действительно, пусть переменная $x$ является функцией от новой переменной $x = x(t)$.

Тогда пользуясь правилом для вычисления дифференциала от произведения, получим:
\[
	d^2y = d(y'(x)dx) = d(y'(x)) \cdot dx + y'(x) \cdot d^2x
\]
\[
	d^2y = y''(x) \cdot dx^2 + y'(x) \cdot x''_t dt^2
\]

Формулы (1) и (2) отличаются вторым слагаемым. Если оно не равно нулю, то свойство инвариантности дифференциала 2-го порядка не выполняется. Второе слагаемое обращается в нуль в том случае, если функция $x(t)$ линейна. Поэтому при линейной замене $x(t) = at + b$ свойство инвариантности дифференциала 2-го (и n-го) порядка верно.




\newpage
\addcontentsline{toc}{subsection}{Теорема Ферма. }
\subsection*{Теорема Ферма.                                                                             }

Пусть выполняются следующие условия:

\begin{enumerate}
	\item Функция определена на промежутке $X$
	\item $x_0$ - внутренняя точка промежутка $X$
	\item функция в точке $x_0$ принимает наибольшее значение, т.е. $f(x) \leq f(x_0)$ для всех точек из промежутка $X$
	\item Существует конечная производная в точке $x_0$
\end{enumerate}
\[ \textbf{Тогда } f'(x_0) = 0 \]

\textit{\textbf{Доказательство:}}

\[
	f'(x_0) = \lim_{\Delta x \rightarrow 0} \frac{f(x_0 + \Delta x) - f(x_0)}{\Delta x} = A
\]

По теории о связи существования предела и односторонних пределов:

\[
	\exists f'(x_0) = A \in \mathbf(R) \Leftrightarrow \exists f'_-(x_0) = f'_+(x_0) = A
\]

\[
	f'_-(x_0) = f'_+(x_0) = \lim_{\Delta x \rightarrow 0 + 0} \frac{f(x_0 + \Delta x) - f(x_0)}{\Delta x} = f'(x_0) = 0
\]
Что и требовалось доказать


\newpage
\addcontentsline{toc}{subsection}{Теорема Ролля. }
\subsection*{Теорема Ролля.                                                                             }

\begin{enumerate}
	\item Функция $f$ определена и непрерывна на отрезке $[a, b]$
	\item Функция $f$ дифференцируема на $(a, b)$
	\item $f(a) = f(b)$
\end{enumerate}

Тогда внутри $(a, b)$ найдется точка, в которой производная функции равна нулю:
\[
	x_0 \in (a, b): f'(x_0) = 0
\]

\textit{\textbf{Доказательство:}}

\textbf{Вторая теорема Вейерштрасса:}

Если функция определена и непрерывна на отрезке, то она достигает на нем своих точных верхней и нижней граней.

Обозначим
\[ M = sup_{x \in [a, b]}f(x); m = inf_{x \in [a, b]} f(x) \]

\begin{enumerate}
	\item $m = M: \forall x_0 \in (a, b): f(x_0) = f(a) = f(b) \Rightarrow f(x) = const \Rightarrow f'(x_0) = 0$
	\item $m < M$
\end{enumerate}

Поскольку функция на концах отрезка принимает одинаковые значения, то одно из значений (либо $m$, либо $M$) достигается во внутренней точке $x_0$.

Тогда для точки $x_0$ выполняются все условия теоремы Ферма, поэтому $f'(x_0) = 0$




\addcontentsline{toc}{subsection}{Теорема Лагранжа. }
\subsection*{Теорема Лагранжа.                                                                          }


\addcontentsline{toc}{subsection}{Теорема Коши. }
\subsection*{Теорема Коши.                                                                              }


\addcontentsline{toc}{subsection}{Правила Лопиталя. }
\subsection*{Правила Лопиталя.                                                                          }


\addcontentsline{toc}{subsection}{Формула Тейлора с различными формами остаточного члена. }
\subsection*{Формула Тейлора с различными формами остаточного члена.                                   }


\addcontentsline{toc}{subsection}{Стандартные разложения по формуле Маклорена. }
\subsection*{Стандартные разложения по формуле Маклорена.                                              }


\addcontentsline{toc}{subsection}{Условия постоянства функции на промежутке.}
\subsection*{Условия постоянства функции на промежутке.                                                }


\addcontentsline{toc}{subsection}{Определение функции, монотонной на промежутке. Критерии строгой и нестрогой монотонности. }
\subsection*{Определение функции, монотонной на промежутке. Критерии строгой и нестрогой монотонности. }


\addcontentsline{toc}{subsection}{Определение точки локального экстремума. Необходимое условие экстремума. Достаточные условия экстремума. }
\subsection*{Определение точки локального экстремума. Необходимое условие экстремума. Достаточные условия экстремума. }


\addcontentsline{toc}{subsection}{Определение выпуклости функции на промежутке. Критерии выпуклости. }
\subsection*{Определение выпуклости функции на промежутке. Критерии выпуклости.                                       }


\addcontentsline{toc}{subsection}{Точки перегиба. Необходимые и достаточные условия точки перегиба.   }
\subsection*{Точки перегиба. Необходимые и достаточные условия точки перегиба.                                        }


\addcontentsline{toc}{subsection}{Асимптоты вертикальные, наклонные и горизонтальные.    }
\subsection*{Асимптоты вертикальные, наклонные и горизонтальные.                                                      }


\addcontentsline{toc}{subsection}{Схема исследования функции и построения ее графика.    }
\subsection*{Схема исследования функции и построения ее графика.                                                      }



\end{document} % конец документа
